%%%%%%%%%%%%%%%%%%%%%%%%%%%%%%%%%%%%%%%%%
% University/School Laboratory Report
% LaTeX Template
% Version 3.1 (25/3/14)
%
% This template has been downloaded from:
% http://www.LaTeXTemplates.com
%
% Original author:
% Linux and Unix Users Group at Virginia Tech Wiki 
% (https://vtluug.org/wiki/Example_LaTeX_chem_lab_report)
%
% License:
% CC BY-NC-SA 3.0 (http://creativecommons.org/licenses/by-nc-sa/3.0/)
%
%%%%%%%%%%%%%%%%%%%%%%%%%%%%%%%%%%%%%%%%%

%----------------------------------------------------------------------------------------
%	PACKAGES AND DOCUMENT CONFIGURATIONS
%----------------------------------------------------------------------------------------

\documentclass[UTF8]{ctexart}

\usepackage{siunitx} % Provides the \SI{}{} and \si{} command for typesetting SI units
\usepackage{graphicx} % Required for the inclusion of images
\usepackage{graphics} % 图片设置
\usepackage{subfigure} 

\usepackage{natbib} % Required to change bibliography style to APA
\usepackage{amsmath} % Required for some math elements 
\usepackage{amssymb} % 使用因为所以符号
\usepackage{fancyhdr} % 使用页眉

\usepackage{algorithm}
\usepackage{algorithmic}
\usepackage{algpseudocode}

\usepackage{listings} % 插入代码
\usepackage{xcolor}

\usepackage{enumerate} % 列表

\floatname{algorithm}{算法}
\renewcommand{\algorithmicrequire}{\textbf{输入:}}
\renewcommand{\algorithmicensure}{\textbf{输出:}}

\lstset{
    %backgroundcolor=\color{red!50!green!50!blue!50},%代码块背景色为浅灰色
    rulesepcolor= \color{gray}, %代码块边框颜色
    breaklines=true,  %代码过长则换行
    numbers=left, %行号在左侧显示
    numberstyle= \small,%行号字体
    %keywordstyle= \color{red},%关键字颜色
    commentstyle=\color{gray}, %注释颜色
    frame=shadowbox%用方框框住代码块
    }

%\usepackage{url} % 引用URL
% \usepackage{cite}
% \newcommand{\upcite}[1]{\textsuperscript{\textsuperscript{\cite{#1}}}} %参考文献上标
%\bibliographystyle{plain}   %引用的样式%

\pagestyle{fancy}
\fancyhf{} 
\cfoot{\thepage} 

\setlength\parindent{0pt} % Removes all indentation from paragraphs

\renewcommand{\labelenumi}{\alph{enumi}.} 

%----------------------------------------------------------------------------------------
%	DOCUMENT INFORMATION
%----------------------------------------------------------------------------------------
\title{算法分析与设计-作业八}

\author{王宸昊 2019214541}

\date{\today}

\begin{document}

\maketitle

%----------------------------------------------------------------------------------------
%	SECTION 1
%----------------------------------------------------------------------------------------

\section{CLRS, Page, 395 24-1}

\subsection{a. 证明无环和拓扑排序顺序}

$G_f$中的边两边的顶点下标是递增的,如果有环路的话,那么一定会出现顶点坐标递减,跟$G_f$边的定义相反,所以$G_f$是无环的。同理$G_b$也是无环的。

根据$G_f$的定义,其所有的顶点都是从低到高的,因此其拓扑排序的顺序就是$<V_1, V_2, \dots, V_{|V|}>$。同理$G_b$

\subsection{b. 证明松弛操作次数}

在最好的情况下,当每个顶点都以正序或倒序的顺序排列时,通过对$E_f$$E_b$中的边进行一次松弛操作就可以得到各个顶点的最短路径。

在最坏的情况下,顶点的顺序先增后减或者先减后增的顺序交替出现,所以整个路径中有$|V|-2$次顺序的改变,每两次顺序改变就要对所有的边进行松弛操作,因此一共进行$|V|/2$遍松弛操作。

\subsection{c. 是否改善了Bellman-Ford算法渐进时间}

并没有降低Bellan-Ford算法的渐进运行时间。只是降低了常数项,而整体的渐进复杂度还是O(VE)


\begin{algorithm}
    \caption{切绳子}
    \begin{algorithmic}[1] %每行显示行号
        \Require $n$绳子的长度
        \Ensure 切割的最大乘积
        \Function{CutRope}{$n$}
            \If{$n == 1 or 2$}
                \Return{$1$}
            \EndIf
            \If{$n == 3$}
                \Return{$2$}
            \EndIf
            \If{$n == 4$}
                \Return{$4$}
            \EndIf
            \State $k \gets 0$
            \For{$i=n$ to $i \leq 4$}
                \State $k \gets k + 1$
                \State $i \gets i - 3$
            \EndFor
            \State $q \gets i$
            \State $res \gets k * 3 * q$
            \State \Return{$res$}
        \EndFunction
    \end{algorithmic}
\end{algorithm}


\begin{displaymath}
    \begin{aligned}
    &\because a+b=n\\
    &\therefore b=n-a>2\\
    &\therefore ab=a(n-a), n>a+2 \\
\end{aligned}
\end{displaymath}
\begin{displaymath}
    \begin{aligned}
    \because ab-(a+b)&=a(n-a)-n \\
    &=na-a^2-n \\
    &=(a-1)n-a^2 \\
    &>(a-1)(a+2)-a^2 \\
    &=a^2+a-2-a^2 \\
    &=a-2 \\
\end{aligned}
\end{displaymath}
\begin{displaymath}
    \begin{aligned}
    &\because a>2 \\
    &\therefore ab-(a+b)>a-2>0 \\
    &\therefore ab>a+b
    \end{aligned}
\end{displaymath}

%----------------------------------------------------------------------------------------
%	SECTION 2
%----------------------------------------------------------------------------------------

\section{CLRS, Page, 370 23.2-7}

\subsection{说明如何在${O^2}$时间内更新传递闭包}

借鉴有向图的闭包:假设添加的边为(u,v),假设添加该边后,顶点x,y可以添加到传递闭包当中,则当且仅当存在边(x, u)和(v,y)时可以添加。
所以依次遍历每一对顶点,判断是否满足上述条件,复杂度为${O^2}$


 \subsection{证明更新传递闭包的复杂度为$\Omega(V^2)$}

 假设现在图G(V,E),随机给所有顶点一个序号,顶点按照递增的顺序排列,每个顶点只与相邻顶点存在一条边,则目前存在|V|-1条边

 假设现在再插入一条末尾指向第一个顶点的边,则每条边都可以抵达其他顶点,则必须通过$n^2 / 2$次更新。

 所以无论什么算法,更新传递闭包的复杂度都是$\Omega(V^2)$。

 \subsection{设计算法任意n次插入序列,算法总时间为$O(V^3)$}

 \begin{lstlisting}
    PASSIVE-CLOUSURE-OF-DYNAMIC-GRAPH(T, u, v)
        let T be |V| * |V| matrixs
        for i = 1 to |V|
            if t[i][u] == 1 and t[i][v] == 0
                for j = 1 to |V|
                    if  t[v][j] == 1 
                        t[i, j] = 1
 \end{lstlisting}

 对于插入边的规模N,最大是$|V^2|$

 对于外层循环需要循环V次,所以插入n次边,总的时间复杂度为$O(V^3)$

 需要注意的是内层循环最多执行$V^2$次,因为再插入多余的边时,内层循环并不执行,由聚合分析可知,插入$V^2$次边,可以在$V^3$内更新到最终的二维闭包矩阵。

\end{document}