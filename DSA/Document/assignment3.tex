\documentclass{article}
% \usepackage[utf8]{inputenc}
\usepackage{xeCJK}
\usepackage[UTF8]{ctex}
\usepackage{algorithm}
%\usepackage{algorithmic}
\usepackage[top=2cm, bottom=2cm, left=2cm, right=2cm]{geometry}
\usepackage{algorithmicx}
\usepackage{algpseudocode}
\usepackage{mathtools}
\usepackage{amssymb}
 
\floatname{algorithm}{算法}
\renewcommand{\algorithmicrequire}{\textbf{输入:}}
\renewcommand{\algorithmicensure}{\textbf{输出:}}

\title{算法设计与分析作业3}
\author{杨树鑫 2019E8013261016}
\date{October 2th , 2019}

\usepackage{natbib}
\usepackage{graphicx}

\begin{document}

\maketitle

本次作业选做3、4题。

%----------------------------------------------------------------------------------------
%	SECTION 1
%----------------------------------------------------------------------------------------

\section{第3题 Cut Rope}
\subsection{算法思路描述}
\par 1. 当$n=1,2,3,4$时,直接给出最优解;
\par 2. 当$n > 4$时,长度为$n$的绳子可以从位置$k$被分为乘积最大的两段$a$和$b$,$a$和$b$又可以按此方法继续各自被分为乘积最大的两段,直到长度小于等于4。(在算法正确性部分将会证明$a>2, b>2,a+b=n$时,有$ab>a+b$成立,因此,对于一个长为$n(n>4)$的绳子,我们总是可以将其继续分为更小的两段,使其乘积比和更大)
\par 再反推,由绳长为$1,2,3,4$只能组成的最大长度为7,观察$5,6,7$这三种情况的最大切法都包含3,而更大的数又可以分成这几个数,所以只需将原绳子先按长度3进行切分,切分到长度小于等于4时停止,即可得到原问题的解。

\subsection{伪代码}
\par 伪代码如算法1所示

\begin{algorithm}[ht]
    \caption{切绳子}
    \begin{algorithmic}[1] %每行显示行号
        \Require $n$绳子的长度
        \Ensure 切割的最大乘积
        \Function{CutRope}{$n$}
            \If{$n == 1 or 2$}
                \Return{$1$}
            \EndIf
            \If{$n == 3$}
                \Return{$2$}
            \EndIf
            \If{$n == 4$}
                \Return{$4$}
            \EndIf
            \State $k \gets 0$
            \For{$i=5$ to $n$}
                \State $OPT(i) \gets 3*OPT(i-3)$
            \EndFor
            \State \Return{$OPT(n)$}
        \EndFunction
    \end{algorithmic}
\end{algorithm}

\subsection{Greedy选择性质和最优子结构}
对一根长为$n(n>4)$的绳子,我们总是可以将其继续分为3和$OPT(n-3)$的两段($OPT(n-3)$表示如果$n-3>4$,那么还继续分割,否则按平凡解给出结果),使其乘积更大,当$n=1,2,3,4$时,可以直接给出最优解。
\begin{center}
    $$OPT(i)=
\begin{cases}
3*OPT(i-3)& \text{i>4}\\
4& \text{i=4}\\
2& \text{i=3}\\
1& \text{i=1or2}\\
\end{cases}$$
\end{center}

\subsection{算法正确性}
首先证明当$a>2, b>2,a+b=n$时,有$ab>a+b$成立
\par 证明:
\begin{displaymath}
    \begin{aligned}
    &\because a+b=n\\
    &\therefore b=n-a>2\\
    &\therefore ab=a(n-a), n>a+2 \\
    &\because ab-(a+b) \\
\end{aligned}
\end{displaymath}
\begin{displaymath}
    \begin{aligned}  
    &=a(n-a)-n \\
    &=na-a^2-n \\
    &=(a-1)n-a^2 \\
    &>(a-1)(a+2)-a^2 \\
    &=a^2+a-2-a^2 \\
    &=a-2 \\
\end{aligned}
\end{displaymath}
\begin{displaymath}
    \begin{aligned}
    &\because a>2 \\
    &\therefore ab-(a+b)>a-2>0 \\
    &\therefore ab>a+b
    \end{aligned}
\end{displaymath}
\par 根据以上证明,当$n=1,2,3,4$时,是平凡的情况,直接给出最优解。
\par 当$n > 4$时,长度为$n$的绳子可以从位置$k$被分为乘积最大的两段$a$和$b$,$a$和$b$又可以按此方法继续各自被分为乘积最大的两段,直到长度小于等于4。再反推,由绳长为$1,2,3,4$只能组成的最大长度为7,观察$5,6,7$这三种情况的最大切法都包含3,而更大的数又可以分成这几个数,所以只需将原绳子先按长度3进行切分,切分到长度小于等于4时停止,即可得到原问题的解。
\par 因此,对一根长为$n(n>4)$的绳子,我们总是可以将其继续分为3和$OPT(n-3)$的两段,使其乘积更大
\par 综上,算法正确

\subsection{算法复杂度}
原问题需要计算$OPT(4) \sim OPT(n)$一遍,所以原问题的时间复杂度为
\begin{center}
    $T(n)$=$O(n)$ 
\end{center} 
\par 由于开了一个$n$的数组存储子问题的最优解,因此原问题的空间复杂度为
\begin{center}
    $O(n)$
\end{center} 

%----------------------------------------------------------------------------------------
%	SECTION 2
%----------------------------------------------------------------------------------------

\section{第4题 Cross River}
\subsection{算法思路描述}
\par 先将整个数组按升序排序,用$i$表示当前未被选取的体重最小的人的索引,用$j$表示当前未被选取的体重最大的人的索引,然后从最大的数开始选取,直到$i=j$:
\par 1. 如果$w[j]+w[i]>limit$,则只选择当前体重最大的人,$j$减小1;
\par 2. 如果$w[j]+w[i] \leq limit$,则选择当前体重最大的人和最小的人,$j$减小1,$i$增大1;

\subsection{伪代码}
\par 伪代码如算法2所示

\begin{algorithm}[ht]
    \caption{过河}
    \begin{algorithmic}[1] %每行显示行号
        \Require $Array$体重数组$w$,$int$人数$n$, $int$最大限制$limit$
        \Ensure 最少的船数量
        \Function{CrossRiver}{$w$, $n$, $limit$}
            \If{$n == 1$}
                \Return{$1$}
            \EndIf
            \State\Call{sort}{$w$}
            \State $i \gets 0, j \gets n-1$
            \State $boat \gets 0$
            \While{$i \leq j$}
                \State $boat \gets boat + 1$
                \If{$w[j]+w[i] \leq limit$}
                    \State $i \gets i+1$
                \EndIf
                \State $j \gets j-1$
            \EndWhile
            \State \Return{$boat$}
        \EndFunction
    \end{algorithmic}
\end{algorithm}

\subsection{Greedy选择性质和最优子结构}
优先选择当前未被选择的最重的人和最轻的人,如果两人体重之和满足限制,就同时载上,否则只载最重的人
\par $w[i]$表示人的体重数组,且已按非降序排序;
\begin{center}
    $$OPT(i,j)=
\begin{cases}
    OPT(i+1, j-1)+1& \text{$w[j]+w[i] \leq limit$}\\
    OPT(i, j-1)+1& \text{$w[j]+w[i] > limit$}\\
\end{cases}$$
\end{center}

\subsection{算法正确性}
1. 对只能载最重的一个人的情况,显然必须要一条船;
\par 2. 对载了最重的一个人的情况下,只能再载一个最轻的人时,即$w[j]+w[i] = limit$时,那就必须载最轻的人才能使船的数量最小。采用反证法
\par 假设不载最轻的人,那么想载两个人的话,重量就会超过限制,必须要再加一条船,与最少船数量矛盾
\par 因此,必须载最轻那个人。
\par 3. 对载了最重的一个人的情况下,还能再载不一定是最轻的人时,即$w[j]+w[i]<limit$,其解不唯一,但解中一定含有第(2)条中的情况,因此为简单起见,就选择(2)的组合方式,即可得到最小船数量。
\par 综上,算法正确


\subsection{算法复杂度}
原问题将数组从两头往中间遍历一遍,所以原问题的时间复杂度为
\begin{center}
    $T(n)$=$O(n)$ 
\end{center} 
\par 由于只用了$int$变量存储结果值,没有数组等其他空间分配,因此原问题的空间复杂度为
\begin{center}
    $O(1)$
\end{center} 

\end{document}
